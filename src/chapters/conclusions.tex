\vspace{30px}\section{Conclusions}
Noticeably, the three NTC plots are the exact inverse of the PTC plots due to their opposite functioning. The current-temperature PTC plot (figure \ref{fig:PTC_curr-temp}) is very similar to the NTC voltage-temperature curve (figure \ref{fig:NTC_volt-temp}) and, seemingly, the current-temperature NTC graph (figure \ref{fig:NTC_curr-temp}) is nearly identical to the PTC voltage-temperature diagram (figure \ref{fig:PTC_volt-temp}). The opposite functioning between the two devices can be observed between the voltage-current characteristic curve of the posistor (figure \ref{fig:PTC_curr-volt}) and of the NTC thermistor (figure \ref{fig:NTC_curr-volt}). The two plots are extremely similar with the only difference that the axises are inverted: in the graph \ref{fig:PTC_curr-volt} the y-axis describes the current meanwhile in the curve \ref{fig:NTC_curr-volt} the y-axis characterizes the voltage.

Nonetheless, both devices are widely used in lots of different scenarios making them extremely versatile: the PTC thermistors find application in scenarios that require self-regulation meanwhile NTC thermistor's sensitivity meets the needs for temperature measurement. 
