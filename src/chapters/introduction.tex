\section{Introduction}
This document aims to give a complete introduction to the world of thermistors. A thermistor, also called \textsl{thermally sensitive resistor}, is a device whose electrical resistivity is strictly bonded with the temperature. More precisely, there are two types of thermistors. The first one is called negative-temperature-coefficient (or \textbf{NTC}), meaning that its resistance changes in the opposite way of temperature: when the temperature rises, the resistance decreases, and vice-versa. The second type is the positive-temperature-coefficient thermistor (also known as \textbf{PTC}) which means that its resistivity increases if the environment gets hotter and is reduced when the temperature decreases.

The first part of the document covers some basic information that's important to know about thermistors, such as the materials and the history of these devices. The second part of the document analyzes the characteristic curves of the PTC and NTC thermistors, exploring the distinct behaviors displayed in different temperature conditions. Furthermore, the document explores practical applications, highlighting the various ways in which these thermistors are applied across different fields.






