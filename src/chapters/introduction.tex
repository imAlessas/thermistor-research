\section{Introduction}
This document aims to give a complete introduction to the world of thermistors. A thermistor, also called \textsl{thermally sensitive resistor}, is a device whose electrical resistivity is strictly bonded with the temperature. More precisely, there are two types of thermistors. The first one is called negative-temperature-coefficient (or \textbf{NTC}), meaning that its resistance changes in the opposite way of temperature: when the temperature raises, the resistance decreases, and vice-versa. The second type is the positive-temperature-coefficient thermistor (also known as \textbf{PTC}) which means that its resistivity increases if the environment gets hotter and is reduced when the temperature decreases.


\todo{Finish introduction.tex}



